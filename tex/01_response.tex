%! TEX root=../main.tex

\section{Responses}
  \subsection{Response 1}
    \begin{quotation}
      The two interpersonal communication skills that I would like to improve
        are assertive self-expression, and being able to adapt my messages to
        account for the unique perspective of the other person.

      Improving my assertive self-expression is important to me because I am not
        really able to express my needs during a one-on-one conversation. I tend
        to think more about the other person and their feelings, rather than my
        own. I have a really hard time saying “no”, making requests, and sharing
        my negative feelings. For those reasons, I do end up getting walked all
        over, in some cases. This is especially prevalent in my work life, as I
        always let people leave before me and end up staying there very late.
        All in all, I am just bad a being honest with my co-workers and boss,
        and expressing my true feelings. I want to be able to stick up for
        myself when I need to. In both my personal and work life I plan on
        gradually adjusting the way I phrase things and how I display them. I
        will practice saying no, rehearsing what it is I want to say, using body
        language, and starting small and working my way up. I know this skill is
        going to take me a while to get good at, so I will have to begin using
        the ideas mentioned before, and start incorporating them in my daily
        life.

      Being able to adapt my messages for each unique individual is something I
        am keen on working on, because it will allow for my message to be
        interpreted as intended. I want to be able to read the person I am
        conversing with, so I can accommodate to their experiences and
        perspectives, to further the discussion. I don’t think I am the worst at
        doing this, but it is definitely something I want to improve on. Some
        preliminary ideas I have about honing in on this skill is to first get
        to know the person. Understanding where people come from and how they
        came to be who they are today, is the key to having a deep and
        meaningful conversation. Analyzing everyone’s personality will also help
        me, there are different approaches to interaction when someone is, let’s
        say, more serious versus some who is more laid back. Improving on both
        these skills is going to be something that is essential in not only my
        personal life, but my work life, and my future as well.
    \end{quotation}

    \paragraph{This is a response to Kristyna Sekera on Post ID 43177567}
      I definitely see how it is difficult to assert your own beliefs with the
        fear of how others will see you once said. It is much easier just to
        ``smile and nod'', so to speak. I saw this everyday when I was still in
        a customer facing role in my industry, where female budtenders where
        subjugated to rude, sexist, and more-often-than-not vulgar, comments. In
        the face of this situation usually another team member on the sales
        floor would find some sort of excuse to remove her from the situation
        and take over the sale, banning customers who failed to comply after
        some arbitrary rule the Human Resources department made up. I believe
        that if more of us were to master this skill we would have a brighter
        perception of our personal quality of life and relationships since not
        every conversation will include ``settling'' without expressing one's
        own opinion.
