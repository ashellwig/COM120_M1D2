\documentclass[stu,12pt]{apa7}
  \usepackage{times}
  \usepackage[english]{babel}
  \usepackage[utf8]{inputenc}
  \usepackage{xcolor}
  \usepackage{hyperref}
  \usepackage{bbding}
  \usepackage{enumitem}
  \usepackage{geometry}
  \usepackage{soul}
  \usepackage{graphicx}
  \usepackage{csquotes}
  \usepackage{bookmark}
  \usepackage{mdframed}
  \usepackage[toc]{appendix}
  % \usepackage[backend=biber,style=apa]{biblatex}
  \usepackage{fancyhdr}
  \usepackage{mdframed}

  % Bibliography Setup
  % \addbibresource{main.bib}

  % Hyperlink Setup
  \hypersetup{
    colorlinks = true,
    urlcolor = blue,
    linkcolor = blue,
    citecolor = blue
  }

  % Page and Text Layout
  \geometry{%
    a4paper,%
    top=1in,%
    bottom=1in,%
    left=1in,%
    right=1in%
  }
  \setlength{\headheight}{15pt}

  % Header
  \lhead{COM120CG1-M1D2}

  % Title Page
  \title{%
    M1D2: Who Do You See in the Mirror
  }
  \shorttitle{Module 1 Discussion 2}
  \author{Ashton Hellwig}
  \authorsaffiliations{Department of Mathematics, Front Range Community College}
  \course{COM115: Interpersonal Communication}
  \professor{Richard Thomas}
  \duedate{November 07, 2020 23:59:59 MDT}
  \date{\today}
  \abstract{%
    \textbf{Overview}\\%
    Identifying our flaws and weaknesses can be hard. Ultimately though, we can
      never improve unless we first figure out what we are doing wrong.\\%

    Earlier this module, you researched effective interpersonal communication
      and created a list of important skills.\\%

    Now, you will analyze your own communication abilities and use this
      information to create a plan to work on the areas in which you need to
      improve. For example, reflect upon King George in the movie clip
      referenced in Module 1 Discussion 1. He is very aware of his stutter,
      which is the first step toward solving the problem.\\%

    This is your opportunity to be honest and open with yourself so you can
      identify interpersonal skills you want to improve and  practice improving
      your interpersonal skills throughout the course. You know what they say:
      practice makes perfect!\\%

    You should spend approximately 3.5 hours on this assignment.
  }

\begin{document}
  % Title Page
  \maketitle

  % Initial Post
  \section{Initial Post}
    \subsection*{Instructions}
      \begin{enumerate}
        \item Using the list of skills of effective interpersonal communicators
          that you developed in Module 1 Discussion 1, rank your current ability
          on each skill.
        \item Use this ranking system to rank your current ability:
          \begin{enumerate}
            \item I never use this skill/I am poor at this skill
            \item Rarely
            \item Sometimes
            \item Most of the time
            \item Always
          \end{enumerate}
        \item Consider the items you ranked as 1, 2, 3, 4, or 5. How important
          are those traits in your life? Why are they important or unimportant?
        \item Be honest with yourself! Only you will see your personal rankings;
          therefore, do not share your rankings in the Main post.
        \item In your Main post, list two (2) interpersonal communication skills
          you will commit to work on improving during the length of our class
          (and beyond). Explain why you have chosen these two interpersonal
          communication skills, using information from your rankings and your
          subjective determination of their importance to you. Include some
          preliminary ideas for how you will improve these traits, both in your
          personal life and professional life (if applicable).
        \item Next, read and comment on the Main post of two classmates. What
          suggestions can you make to help them achieve their goals?
          \begin{itemize}
            \item Remember that these feedback posts should contain useful and
              friendly suggestions using a positive communication tone; you are
              not attempting to “fix” anything for your classmates.
            \item Use “I” statements rather than “You” statements. For example:
              ``I have found that talking in a calm voice is essential to
              resolving conflict''.
            \item Be respectful! Follow the Golden Rule by treating others the
              way you would like to be treated.
            \item Remember to read and fully comply with the Netiquette Guide
              found in the Syllabus section.
          \end{itemize}
        \item Now, go out and have fun while practicing improvement of the two
          interpersonal communication skills you chose to improve. IMPORTANT
          NOTE:\@ It is very important to keep track of what you practice, when
          you practice (time and date), who you practiced with, what strategies
          you used when you practice, and the outcome of that practice. Keep
          track in a Word document. Why? Next week you will begin adding your
          practice results, saved in a Word document, as an attachment to the
          Main Post of Discussion 2 Assignments. I enthusiastically look forward
          to reviewing your practice results appearing in the Word document.
          Appearing below is a link to a sample of how to format and enter
          entries in a practice log. The practice will be updated each week!
      \end{enumerate}

    % Introduction
    \newpage
    \subsection{How I rank Myself With Previously Mentioned Interpersonal %
      communication skills}
      In the last discussion I had trouble comming up with five interpersonal
        communication skills, and thus I only had three. Below is how I would
        rank myself against these three skills as described by the discussion
        prompt.

      \begin{enumerate}
        \item \textit{A Strong Hold on Psychological Context}
          \begin{itemize}
            \item[\XSolidBrush] I never use this skill/I am poor at this skill
            \item[\XSolidBrush] Rarely
            \item[\XSolidBrush] Sometimes
            \item[\CheckmarkBold] Most of the time
            \item[\XSolidBrush] Always
          \end{itemize}
        \item \textit{%
          Enter Every Conversation with the intention to understand, rather %
            than with the intention to reply%
        }
          \begin{itemize}
            \item[\XSolidBrush] I never use this skill/I am poor at this skill
            \item[\CheckmarkBold] Rarely
            \item[\XSolidBrush] Sometimes
            \item[\XSolidBrush] Most of the time
            \item[\XSolidBrush] Always
          \end{itemize}
        \item Effectively Utilize Silence
          \begin{itemize}
            \item[\XSolidBrush] I never use this skill/I am poor at this skill
            \item[\XSolidBrush] Rarely
            \item[\CheckmarkBold] Sometimes
            \item[\XSolidBrush] Most of the time
            \item[\XSolidBrush] Always
          \end{itemize}
      \end{enumerate}

    % Skills I wish to Improve
    \subsection{Personally Selected Skills for Improvement}
      I chose to work on the skills involving a stronger hold on psychological
        context as well as listening with more of an intent to understand,
        rather than with the intent to respond. I believe these are both very
        important skills I tend to lack more-often-than-not. I also am fully
        away that I tend to try to seem more like an expert on everything I talk
        about, rather than trying to learn instead. I think it's because rather
        than viewing the conversation as an educational opportunity I would
        view it in a more ``business-connection'' focused sense. I also tend to
        say more than is needed, whereas the Ted Talked we viewed described the
        preference of being more consise.

      I feel the best way to improve these skills in which I lack would be to
        simply \emph{talk less}. This also goes as far as to stop immediatly
        thinking of how what someone is telling me relates to my own life and
        ask questions even if I already have the prior knowledge as someone may
        have some that I do not.
  % Replies
  % %! TEX root='..\main.tex'

\part{}

\end{document}
