\documentclass[stu,12pt]{apa7}
  \usepackage{times}
  \usepackage[english]{babel}
  \usepackage[utf8]{inputenc}
  \usepackage{xcolor}
  \usepackage{hyperref}
  \usepackage{bbding}
  \usepackage{enumitem}
  \usepackage{geometry}
  \usepackage{soul}
  \usepackage{graphicx}
  \usepackage{csquotes}
  \usepackage{bookmark}
  \usepackage{mdframed}
  \usepackage[toc]{appendix}
  % \usepackage[backend=biber,style=apa]{biblatex}
  \usepackage{fancyhdr}
  \usepackage{mdframed}

  % Bibliography Setup
  % \addbibresource{main.bib}

  % Hyperlink Setup
  \hypersetup{
    colorlinks = true,
    urlcolor = blue,
    linkcolor = blue,
    citecolor = blue
  }

  % Page and Text Layout
  \geometry{%
    a4paper,%
    top=1in,%
    bottom=1in,%
    left=1in,%
    right=1in%
  }
  \setlength{\headheight}{15pt}

  % Header
  \lhead{COM120CG1-M1D2}

  % Title Page
  \title{%
    M1D2: Who Do You See in the Mirror
  }
  \shorttitle{Module 1 Discussion 2}
  \author{Ashton Hellwig}
  \authorsaffiliations{Department of Mathematics, Front Range Community College}
  \course{COM115: Interpersonal Communication}
  \professor{Richard Thomas}
  \duedate{November 07, 2020 23:59:59 MDT}
  \date{\today}
  \abstract{%
    \textbf{Overview}\\%
    Identifying our flaws and weaknesses can be hard. Ultimately though, we can
      never improve unless we first figure out what we are doing wrong.\\%

    Earlier this module, you researched effective interpersonal communication
      and created a list of important skills.\\%

    Now, you will analyze your own communication abilities and use this
      information to create a plan to work on the areas in which you need to
      improve. For example, reflect upon King George in the movie clip
      referenced in Module 1 Discussion 1. He is very aware of his stutter,
      which is the first step toward solving the problem.\\%

    This is your opportunity to be honest and open with yourself so you can
      identify interpersonal skills you want to improve and  practice improving
      your interpersonal skills throughout the course. You know what they say:
      practice makes perfect!\\%

    You should spend approximately 3.5 hours on this assignment.
  }

\begin{document}
  % Title Page
  \maketitle

  % Initial Post
  \section{Initial Post}
    \subsection*{Instructions}
      \begin{enumerate}
        \item Using the list of skills of effective interpersonal communicators
          that you developed in Module 1 Discussion 1, rank your current ability
          on each skill.
        \item Use this ranking system to rank your current ability:
          \begin{enumerate}
            \item I never use this skill/I am poor at this skill
            \item Rarely
            \item Sometimes
            \item Most of the time
            \item Always
          \end{enumerate}
        \item Consider the items you ranked as 1, 2, 3, 4, or 5. How important
          are those traits in your life? Why are they important or unimportant?
        \item Be honest with yourself! Only you will see your personal rankings;
          therefore, do not share your rankings in the Main post.
        \item In your Main post, list two (2) interpersonal communication skills
          you will commit to work on improving during the length of our class
          (and beyond). Explain why you have chosen these two interpersonal
          communication skills, using information from your rankings and your
          subjective determination of their importance to you. Include some
          preliminary ideas for how you will improve these traits, both in your
          personal life and professional life (if applicable).
        \item Next, read and comment on the Main post of two classmates. What
          suggestions can you make to help them achieve their goals?
          \begin{itemize}
            \item Remember that these feedback posts should contain useful and
              friendly suggestions using a positive communication tone; you are
              not attempting to “fix” anything for your classmates.
            \item Use “I” statements rather than “You” statements. For example:
              ``I have found that talking in a calm voice is essential to
              resolving conflict''.
            \item Be respectful! Follow the Golden Rule by treating others the
              way you would like to be treated.
            \item Remember to read and fully comply with the Netiquette Guide
              found in the Syllabus section.
          \end{itemize}
        \item Now, go out and have fun while practicing improvement of the two
          interpersonal communication skills you chose to improve. IMPORTANT
          NOTE:\@ It is very important to keep track of what you practice, when
          you practice (time and date), who you practiced with, what strategies
          you used when you practice, and the outcome of that practice. Keep
          track in a Word document. Why? Next week you will begin adding your
          practice results, saved in a Word document, as an attachment to the
          Main Post of Discussion 2 Assignments. I enthusiastically look forward
          to reviewing your practice results appearing in the Word document.
          Appearing below is a link to a sample of how to format and enter
          entries in a practice log. The practice will be updated each week!
      \end{enumerate}

    % Introduction
    \newpage
    \subsection{How I rank Myself With Previously Mentioned Interpersonal %
      communication skills}
      In the last discussion I had trouble coming up with five interpersonal
        communication skills, and thus I only had three. Below is how I would
        rank myself against these three skills as described by the discussion
        prompt.

      \begin{enumerate}
        \item \textit{A Strong Hold on Psychological Context}
          \begin{itemize}
            \item[\XSolidBrush] I never use this skill/I am poor at this skill
            \item[\XSolidBrush] Rarely
            \item[\XSolidBrush] Sometimes
            \item[\CheckmarkBold] Most of the time
            \item[\XSolidBrush] Always
          \end{itemize}
        \item \textit{%
          Enter Every Conversation with the intention to understand, rather %
            than with the intention to reply%
        }
          \begin{itemize}
            \item[\XSolidBrush] I never use this skill/I am poor at this skill
            \item[\CheckmarkBold] Rarely
            \item[\XSolidBrush] Sometimes
            \item[\XSolidBrush] Most of the time
            \item[\XSolidBrush] Always
          \end{itemize}
        \item Effectively Utilize Silence
          \begin{itemize}
            \item[\XSolidBrush] I never use this skill/I am poor at this skill
            \item[\XSolidBrush] Rarely
            \item[\CheckmarkBold] Sometimes
            \item[\XSolidBrush] Most of the time
            \item[\XSolidBrush] Always
          \end{itemize}
      \end{enumerate}

    % Skills I wish to Improve
    \subsection{Personally Selected Skills for Improvement}
      I chose to work on the skills involving a stronger hold on psychological
        context as well as listening with more of an intent to understand,
        rather than with the intent to respond. I believe these are both very
        important skills I tend to lack more-often-than-not. I also am fully
        away that I tend to try to seem more like an expert on everything I talk
        about, rather than trying to learn instead. I think it's because rather
        than viewing the conversation as an educational opportunity I would
        view it in a more ``business-connection'' focused sense. I also tend to
        say more than is needed, whereas the Ted Talked we viewed described the
        preference of being more concise.

      I feel the best way to improve these skills in which I lack would be to
        simply \emph{talk less}. This also goes as far as to stop immediately
        thinking of how what someone is telling me relates to my own life and
        ask questions even if I already have the prior knowledge as someone may
        have some that I do not.


  % Replies
  %! TEX root=../main.tex

\section{Responses}
  \subsection{Response 1}
    \begin{quotation}
      The two interpersonal communication skills that I would like to improve
        are assertive self-expression, and being able to adapt my messages to
        account for the unique perspective of the other person.

      Improving my assertive self-expression is important to me because I am not
        really able to express my needs during a one-on-one conversation. I tend
        to think more about the other person and their feelings, rather than my
        own. I have a really hard time saying “no”, making requests, and sharing
        my negative feelings. For those reasons, I do end up getting walked all
        over, in some cases. This is especially prevalent in my work life, as I
        always let people leave before me and end up staying there very late.
        All in all, I am just bad a being honest with my co-workers and boss,
        and expressing my true feelings. I want to be able to stick up for
        myself when I need to. In both my personal and work life I plan on
        gradually adjusting the way I phrase things and how I display them. I
        will practice saying no, rehearsing what it is I want to say, using body
        language, and starting small and working my way up. I know this skill is
        going to take me a while to get good at, so I will have to begin using
        the ideas mentioned before, and start incorporating them in my daily
        life.

      Being able to adapt my messages for each unique individual is something I
        am keen on working on, because it will allow for my message to be
        interpreted as intended. I want to be able to read the person I am
        conversing with, so I can accommodate to their experiences and
        perspectives, to further the discussion. I don’t think I am the worst at
        doing this, but it is definitely something I want to improve on. Some
        preliminary ideas I have about honing in on this skill is to first get
        to know the person. Understanding where people come from and how they
        came to be who they are today, is the key to having a deep and
        meaningful conversation. Analyzing everyone’s personality will also help
        me, there are different approaches to interaction when someone is, let’s
        say, more serious versus some who is more laid back. Improving on both
        these skills is going to be something that is essential in not only my
        personal life, but my work life, and my future as well.
    \end{quotation}

    \paragraph{This is a response to Kristyna Sekera on Post ID 43177567}
      I definitely see how it is difficult to assert your own beliefs with the
        fear of how others will see you once said. It is much easier just to
        ``smile and nod'', so to speak. I saw this everyday when I was still in
        a customer facing role in my industry, where female budtenders where
        subjugated to rude, sexist, and more-often-than-not vulgar, comments. In
        the face of this situation usually another team member on the sales
        floor would find some sort of excuse to remove her from the situation
        and take over the sale, banning customers who failed to comply after
        some arbitrary rule the Human Resources department made up. I believe
        that if more of us were to master this skill we would have a brighter
        perception of our personal quality of life and relationships since not
        every conversation will include ``settling'' without expressing one's
        own opinion.

  %! TEX root=../main.tex

\section{Responses}
  \subsection{Response 2}
    \begin{quotation}
      The reason for taking this class was to see where my communication flaws
        were, and I don't like what they are. However, I want to advance in my
        career. I want to be closer to my son, and I want people to know it's
        easy to communicate with me. I was a teacher for 14 years, and it was
        easy because I taught a subject, Life and Physical Science, I love. It
        was easy for me, but it may not have been easy for my students. Now, I
        teach new employees on the unit. I always have great reviews, but not so
        great reviews as a coworker. I have a great quality which is being
        direct,but my delivery needs work. For instance, my delivery could use
        more and better eye contact with direct communication. I have a tendency
        to show my emotions in my face that are not favorable, and I'm not proud
        of it. For instance, the charge nurse is new and unfamiliar with things.
        Another nurse called to ask her for help calling a doctor, and the
        charge nurse said she would help. The charge nurse hung up the phone,
        and stated that she didn't know how to do it. I happened to be near,
        turned to her, and gave a face that stated ``what''. The charge nurse
        looked down. I felt bad. So I apologized for the look on my face, and
        stated I would help her know how to call a doctor. Later, we talked
        about how that situation was never brought to her before, but everyone
        on the floor needs to show compassion and patience with her. Afterwards,
        I went to the nurse that called her, and I instructed her what to do as
        well. It was a simple solution in the end. So instead of frowning or
        making a face at someone, maybe I can ask an open ended question to
        understand the situation better, and then I can have input on how to
        make things better.

      I carry a list with me almost always. It may be a list of groceries, a
        to-do list around the house, or a to-do list for work. It's how I keep
        my thoughts organized. For the items on the list, I can multitask those
        to complete duties assigned. When someone wants to add to the list while
        I'm multitasking, I'm less receptive. I say sometimes,
        ``walk and talk''. It reads rude as I write it, but at the time it's
        happening it doesn't. I want to learn to stop for just a second to
        listen, but I feel I will get out of rhythm and things will get chaotic.
        Things won't get chaotic, but I still feel that way. When a coworker and
        my brother stops to listen to me talk, it seems so easy and effortless,
        and it makes me feel good. Respected. They are confident, and I can
        tell. Hence, I have a plan. If I'm in the zone, I can ask the person to
        wait just a second so I can stop, grab paper and pencil, and begin to
        listen.

      Learning these communicative skills will definitely put me at the top,
        and, undoubtedly, be the hardest things I will ever learn to do in five
        weeks.
    \end{quotation}

    \paragraph{This is a response to Diata Hart on Post ID 43188941}
      Placeholder.

\end{document}
